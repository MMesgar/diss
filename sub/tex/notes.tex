#####################################################################################
\cite{todirascu16}
#####################################################################################

Although psycholinguistic experiments have shown that a higher level of cohesion and coherence between a pair of
related sentences decreases their reading time (Kintsch et al., 1975; Mason and Just, 2004), the added
value of these textual dimensions for readability models (compared to traditional features) remains unclear, 

Coherence is defined as a “semantic property of discourse, based on the interpretation of each individual
sentence relative to the interpretation of other sentences” (Van Dijk, 1977, 93)

Cohesion is a property of text represented by explicit formal grammatical ties (discourse connectives)
and lexical ties that signal how utterances or larger text parts are related to each other.

In this article, we consider explicit lexical ties such as anaphoric, co-reference and lexical chains as
cohesive features. We study the correlation between these cohesive features and text complexity


#####################################################################################
\cite{petersencasper15}
#####################################################################################

Good definition of coherence and three levels of coherence: intentional, attentional, organizational 

Good related work
The idea of more entity in a sentence meaning less coherent text

Conveying the message that outdegree is not the best for the entity graph. They computed other graph properties.

Evaluation on 
	the sentence ordering task
	Information Retrieval task (TREC): as post processing re-rank the retrieved documents

Based on this paper: our model is a coherence model that captures local and global coherence.

Entity-based models are based on the Centering theory.

The model is unsupervised.

Idea: using information entropy: information entropy is the expected value of the information content of a random variable. coherence score is the inverse of the entropy of entities

A connection between probability of the entities and normalization method of the normalized entity graph. Another connection is Entity distance and Adjacent Topic flow.

Check the highlighted references for further readings; they are good papers for coherence.

#####################################################################################
 \cite{} -- Micha Elsner's Thesis
#####################################################################################
At a high level, a coherent document discusses a sequence of topics in a structured way, with each topic
tending to occupy a single segment of the text (Hearst, 1997; Galley et al., 2003).


Inside each segment, adjacent sentences link to one another in a local structure, in which they share
not just a general topic but some specific logical relationship.

Local coherence models fall into several types: rhetorical models try to explicitly describe the way propositions link up to form an argument or narrative. 
Lexical models capture the tendency for adjacent sentences to use similar vocabulary, since they
express related propositions. 
Entity-based models look at the way entities objects existing in the world are
mentioned in the discourse.

Marcu (1997) and Mellish et al. (1998), Both these papers focus on text generation; they take
input in a non-linguistic format where the relationships between propositions are made explicit. 
For texts that may or may not be coherent in the first place (such as those formed
by extracting sentences from a multidocument collection), this problem is likely to be even worse. Pitler
and Nenkova (2008) obtain some use from rhetorical structure in predicting readability for human-authored
documents where rhetorical relations were marked by hand, suggesting that this type of information would
indeed be useful if it were available.

Lexical features, on the other hand, require no annotation at all; they are accessible directly from the text.
However, their expressive power is limited; in general, lexical methods aim to measure similarity between
pairs of sentences adjacent in a text, with little ability to predict the direction of the relationship or how it
relates to a higher-level structure. Basic lexical methods restrict themselves to counting repeated words; a
variety of metrics are covered in Lapata and Barzilay (2005). Information retrieval techniques, most notably
TF-IDF (Jones, 1972), can be used to approximate the importance of each word. More sophisticated models
aim to learn associations between different words (for instance, “car” is a good context for “tire”, or “brakes”
(Prince, 1981)). Foltz et al. (1998) uses a separate dimensionality reduction phase, LSA (Deerwester et al.,
1990) to learn words that are similar to one another. Lapata (2003) is the rst to learn word associations for
local coherence directly, in a manner that respects ordering (so it is able to learn that ``tire" is not as good a
context for ''car" as vice versa). Soricut and Marcu (2006) uses IBM model 1 (Brown et al., 1993) to learn
these word associations.
Work on lexical chains searches for larger-scale groups of related words which span multiple sentences;
rather than relating each sentence to the next, they build larger structures, “chains” of related words which
persist throughout a topical segment. As with other lexical methods, lexical chains require some basic method
for finding significant repeated or similar words. Galley et al. (2003) uses TF-IDF; Morris and Hirst (1991)
describes a variety of methods based on thesauri.

The original motivation for computational coherence modeling was text planning; Marcu (1997) points out
that a system attempting to explain a set of facts would require a way of ordering those facts, and that this
ordering should result in coherent transitions between adjacent units of text.
Jing and McKeown
(1999) nds that human summarizers often reorder sentences to create more coherent output, suggesting that
poor ordering can compromise readers' understanding; Barzilay et al. (2002) demonstrates this empirically,
as does Lapata (2006). Barzilay et al. (2002) also observes that imposing a coherent ordering on the resulting
set of sentences is non-trivial, motivating the use of coherence models to solve the problem. 

Barzilay et al. (2002) use chronological information and topical clustering; Lapata (2003) learns word-to-word associations that typify coherent transitions;
Lapata and Barzilay (2005) and Barzilay and Lapata (2005) use a syntactic model of information structure.

In addition to text ordering, local coherence models have also been used to score the uency of texts 
written by humans or produced by machine. Evaluation of human-authored text can characterize how difcult
it is to read. The rst use of coherence modeling in this context is Foltz et al. (1998), who use similarity
between adjacent sentences to measure readability of encyclopedia articles and a historical text. 

Pitler and Nenkova (2008) models readers' reported difculty in reading WSJ news articles. 
Scores for
human-authored text are also useful in an educational context, since they can augment human grading for
student essays. Miltsakaki and Kukich (2004), Higgins et al. (2004) and Burstein et al. (2010) model various
aspects of coherence and nd that they are predictive of the grades eventually assigned.
coherence modeling is an implicit ingredient in many attempts to understand the hidden structure
of text. 

%This article focuses on
%local coherence, which captures text relatedness at the level of sentence-to sentence transitions. 
%Local coherence is undoubtedly necessary for global coherence and has received considerable attention in computational linguistics (Foltz,
%Kintsch, and Landauer 1998; Marcu 2000; Lapata 2003; Althaus, Karamanis, and Koller2004;Karamanisetal.2004

%The entity grid, however, does not commit itself to fine-grained rankings of the centers or to a pre-specified transition rule. Instead, it provides a representation which can be constructed from relatively light-weight syntactic analyses, and in which the kinds of rules specified by the theory will be efficiently learnable.

%Lapata (2003) gives the first model for learning word associations for coherence from data.
%Her model gives the probability of a sentence W with words $w_i$ following a sentence V with words $V_j$ as:
%$P(W) = \prod_{i} \prod_{j}p(w_i|v_j)$
%This model explains each word w as produced by a predecessor word v, although which word in the previous
%sentence should serve as predecessor for each w is unknown.
%
%Soricut and Marcu (2006), inspired by a personal communication from Kevin Knight, use IBM model 1
%(Brown et al., 1993) to learn word-to-word associations.
%
%IBM uses hidden variables to obtain a true generative
%model with the same lexical parameterization as Lapata (2003).
%
%IBM solves the problem that we do not know the correct predecessor word vj for each wi by introducing
%a hidden alignment variable ai 2 [0::j] which indicates precisely which previous word is responsible for each
%current word.
%
%Because IBM-1 has hidden variables, its parameters cannot be learned by direct estimation; conventionally,
%EM or variational EM are used instead.
%
%Elsner's implementation of IBM-1 produces and conditions on nouns and verbs only;
%
#####################################################################################
\cite{Barzilay} Modeling Local Coherence: An Entity-Based Approach
#####################################################################################
Linguistic Modeling.
Entity-based accounts of local coherence have a long tradition
within the linguistic and cognitive science literature (Kuno 1972; Chafe 1976; Halliday
and Hasan 1976; Karttunen 1976; Clark and Haviland 1977; Prince 1981; Grosz, Joshi,
and Weinstein 1995). A unifying assumption underlying different approaches is that
discourse coherence is achieved in view of the way discourse entities are introduced
and discussed.This observation is commonly formalized by devising constraints on the
linguistic realization and distribution of discourse entities in coherent texts.


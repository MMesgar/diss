\chapter{Introduction}
\label{chapt:intro}
\begin{quotation}
 

\textbf{A Definition of Text.}
Having a definition of text is crucial because it contains basic assumptions made by researcher, many studies suffer for the lack of it. 
``text" is defined from its different perspectives \cite{kantor77,simmons79,bamberg83}.
One of the most carefully designed definitions of text is that of \newcite{haliday76} that describe a text as a passage of discourse which is:
\begin{itemize}
\item a semantic, not a grammatical unit
\item encoded in sentences, but nor structurally related to them
\item related to context of situation by a consistency if register
\item reasonably homogeneous and thus consistent across all texts
\item characterized by certain linguistic features which are the basic of cohesion in the text and thereby give it texture
\item not defined by size.
\end{itemize}

We use this definition of text in this study as well. 

\textbf{Refer to Chapter 2 of \newcite{stoddard91} to see her explanation of the cohesion and coherence that are not distinguishable and can be used interchangeably. }



As most literature in text linguistics argues (Halliday and Hassan 1976, Lyons 1981, De Beaugrande and Dressler 1981, inter alia), a text processes coherence which is to say that the content is organized in a way that is easy for humans to read and understand. 

Tangkiengsirisin (Cohesion and coherence in Text) look at different aspects of coherence and cohesion of texts written in English as two essential elements that facilitate textual continuity. 
The terms of coherence and cohesion is defined differently by different linguistics. 
For some, the two terms are interchangeable or imply each other, and for others they are independent of one another. 
The concept of cohesion was introduced by Halliday and Hasan (1976) whose major concern is to investigate how sentences are linked together in a text. 
Elements of sentences that link sentences are called cohesive ties in Halliday and Hasan (1976). 
According to Haliday and Hasan (1976), the writer is able to hold together meanings in the related sentences in a number of ways, and cohesion is created to establish the structure of meaning. 
They also claim that cohesion is a factor that indicates if a text is well-connected or merely is a group of unrelated sentences. 
Halliday and Hasan (1976) explicitly express that cohesion does not deal with content of a text
``cohesion does not concern with what a text means, it concerns how the text is constructed as a scientific edifice". 
Discourse involves the context and need to be interpreted through the understanding of discourse structures and the use of many strategies; for example, two comprehend discourse, we interpret the discourse assuming that if one thing said after another, the two things are related in some way. 
Coherence can be regarded as a connection between utterances with discourse structure, meaning, and action being combined (Schiffrin, 1987). 
To Schiffrin, cohesive devices are clues that help locate meanings and accommodate the understanding of a conversation. 
Discourse coherence, therefore, is dependent on a speaker's successful integration of different verbal and nonverbal devices to situate a message in an interpretive frame and a hearer's corresponding synthetic ability to interpret such cues as a totality in order to interpret that message. 
Coherence may be treated as `` semantic property of discourse, based on the interpretation each individual sentence relative to the interpretation of other sentences" (Van Dijk, 1977: p.93). 
Coherence between sentences, in Van Dijk's point of view, is ``based not only the sequential relation between expressed and interpolated propositions, but also on the topic of discourse of a particular passage."
The two levels of coherence are micro-coherence, which is the linear of sequential relations between propositions, and the macro-coherence, the global or overall coherence of a discourse in terms of hierarchal topic progression. 
A text must have surface cohesion as well as overall coherence, and sentences in a coherent text must ``conform to the picture of one possible world in the experience or imagination of the receiver" (Enkvist, 1978, p128).
A message also must provide adequate signals for the listener or the readers to make connections for understanding of a text. 
Lovejoy and Lance (1991), in their study of written discourse, show that cohesion can be achieved through the operation of theme-rheme. 
This movement represents how information is managed.
According to Lovejoy and Lance (1991), theme is ``the point of departure" for the representation of information" and rheme ``constitutes the information the writer wishes to impact about the theme".
These two elements are presented alternatively in a text to form a connected text. 
while theme conveys information that is initially introduced in discourse, rheme presents specific information regarding the theme. 
As this movement continues, ideas in a text or discourse are expected to flow along smoothly and are easier for the reader to understand. 
While old information (theme) is presented as background information in each statement, new information (rheme) is introduced to clarify information in the theme. 
Morgen and Sellner (1980) emphasize that the cohesion and coherence are independent.  
Carrel (1982) contends that cohesion does not bring about coherence, but believes that the cohesion is the effect of coherence not a cause of coherence. 
From a textual perspective, Hoey (1991) examined how lexical cohesive elements would make a text organized. 
He examined how lexical features and syntactic repetition would contribute to cohesion. 
Within this general framework, cohesion is regarded as an element that accommodates coherence. 
When a text is cohesive and coherence, it will enable the reader to process information more rapidly. 
Hoey claims that ``cohesion is a property of the text and coherence is a facet of the reader's evaluation of a text". 
According to Hoey (1991) lexical connections as a major cohesive device constructs a matrix and creates a net of bonds in texts. 
He proposes that lexical relations can show the relatedness of sentences within a text. 
Johns (1986) divides coherence into two types: text-based and reader-based. 
By her definition, text-based coherence refers to an inherent features of the text, which involves cohesion and unity. 
This type of coherence involved how sentences are linked and how text is unified. 
Reader-based coherence on the other hand requires successful interaction between the reader and the text. 
In this type, coherence is based on the degree of compatibility between the reader's expectations and the intended meaning through the underlying structure of a text. 
Connor and Johns (1990) describe coherent text ``as text in which the expectations of the reader are fulfilled".
Readers use her knowledge of the world to interprets a text expecting that their knowledge will correspond to the organization and argument of a text. 
The reader relies on this kind of knowledge to anticipate information that will be subsequently presented. 
Interacting with readers, a coherent text accommodates the readers expectations of the order of ideas, contributing to the comprehension of the text. 
By the same token, as logical ideas are presented through well-connected words and sentence the writer helps readers interpret and process information in a text more easily (Tannen, 1984).
Lautamatti (1987) defines the term topic as what the sentence is about and the term comment as information about the topic. 
All sentence topics are related in certain ways to the global discourse topic of the text. 
The patterns of relations between discourse topics and subtopics are called topical development of discourse. 
Grabe (1985) proposes the pragmatic function of coherence. 
He identifies three features that are essential to coherence: a discourse theme, a set of relevant assertions relating logically among themselves. 
The sentence topic in English is often correlates with grammatical subject and comment often correlates with the grammatical predicate, which bears the sentinel focus. 
While patterns of theme and rheme connections can account for only some part of a text, diversity of patterns deal with an entire text. 
The theory of cohesive ties introduced by Halliday and Hasan (1976) was modified into a theory of cohesive harmony (Hasan 1984, Haliday and Hasan, 1989). 
Due to the limitations of the use of cohesive ties to analyze texts as coherent and well-written, Hasan (1984) formulated a new theory to account for the fact that cohesion contributes to coherence. 
In her new approach, coherence is not determined by the type and quantity of cohesive ties that appear in a text, but it is mainly characterized by the degree and frequency of with which these ties interact with each other. 
According to this theory, there are two cohesive ties which can interact with each others: those that form identity chain expressed with the use of co-referent entities, and those that form similarity strings expressed through substitution, ellipsis, repetition, synonymy, antonymy, hyponymy, and meronym. 
Interaction does occur when one member of a string or chain is in the identical relationship to more than one member of another string or chain.

An increasing number of researchers and practitioners in NLP face the prospect of havin to work with entire texts, rather than individual sentences. 
While it is clear that text must have useful structure, its nature is less clear, making it more difficult to exploit in applications. 
Discourse commonly comprises a sequences of sentence although it can be found even within a single sentence. 
Within a discourse, the patterns formed by its sentences mean that the whole conveys more than the sum of its separate parts. 
Another point  about discourse is that it exploits language  features, which allow speaker to specify that they are talking about something they have talked about before in the same discourse; indicating a realtion that holds between the states, events, beliefs, etc.  presented in the discourse; changing to a new topic or resuming one from earlier in the discourse.
Discourse structures are the patterns that one sees in multi-sentence (multi-clausal) texts. 
Recognizing these patterns in terms of the elements that compose them is essential to correctly deriving and interpreting information in the text. 
The elements may be topic, each  about a set entities and what is being said about them.
These type of relations are mainly called as coherence relations by Webber et al. (NLE 2012). 
Discourse can be structured by its topics, each comprising a set of entities and a limited range of things being said about them. 
Each topic may involve a set of entities, which may (but not have to) change from topic to topic. 
This aspect of structure has been modeled as entity chains (Barzilay and Lapata, 2008).
Patterns of entity chains can also be characteristic of particularly types of discourse, and therefore be of value in assessing the quality of automatically generated text. 
Low-level evidence for the topic structure of discourse comes from the strong correlation between topic and lexical usage, which Halliday and Hasan (1976) call lexical cohesion. 
Lexical chains can simply be a matter of the density of related terms within a segment, or of particular patterns of related terms, such as lexical chains (Barzilay and Elhadad 1997; Galley et al, 2003; Clarke and Lapata, 2010) defined as sequence of semantically related words. 


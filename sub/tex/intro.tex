\chapter{Introduction}
\label{chapt:intro}
\begin{quotation}
 

\textbf{A Definition of Text.}
Having a definition of text is crucial because it contains basic assumptions made by researcher, many studies suffer for the lack of it. 
``text" is defined from its different perspectives \cite{kantor77,simmons79,bamberg83}.
One of the most carefully designed definitions of text is that of \newcite{haliday76} that describe a text as a passage of discourse which is:
\begin{itemize}
\item a semantic, not a grammatical unit
\item encoded in sentences, but nor structurally related to them
\item related to context of situation by a consistency if register
\item reasonably homogeneous and thus consistent across all texts
\item characterized by certain linguistic features which are the basic of cohesion in the text and thereby give it texture
\item not defined by size.
\end{itemize}

We use this definition of text in this study as well. 

\textbf{Refer to Chapter 2 of \newcite{stoddard91} to see her explanation of the cohesion and coherence that are not distinguishable and can be used interchangeably. }





% for sublime text 3
%!TEX root = diss.tex

\chapter{Coherence Modeling}
\label{ch:coherence}

Local coherence modeling is a crucial task for natural language processing with varying specifications over the years. 
This chapter gives related linguistic definitions, which are related to the task that is tackled in this thesis. 


\section{Problem Definition}
\label{sec:coh-def}

In this thesis, we tackle the problem of local coherence modeling. 
The popular definition of this task is to model how sentences in a text are related one another. 
This task has been the focus of the majority work in text processing (see Section \ref{sec:related_work}). 
Variations of the task consider different types of relations, such as rhetorical \cite{hovyeduard89} or lexical \cite{morris91}, between different spans of texts, such as clauses \cite{strube.col98} or sentence \cite{halliday76}, in various text types, such as dialogue \cite{wangxinhao13} or monologue \cite{barzilay08}. 
Here we give a formal definition of this problem, which will be used in the following chapters of this thesis. 

\subsection{Formal Modeling}

In order to provide a formal definition of the task, i.e.\ local coherence modeling in texts, we first need to give a definition of what we refer to as a text. 
In this thesis, the word text refers to formal written and monologue sequence of sentences that transmit a meaning as a whole.  
We assume that a text consists of more than one sentence. 

\begin{definition}
Text $T$ is a sequence of finite number of sentences $[s_0, s_1, s_2, ..., s_n]$ where the number of sentences is greater than 1, $n>0$.   
\end{definition}

Each sentence in the above definition of a text is a list of words that form a sentence structure. 

\begin{definition}
Sentence $s_i$ is a sequence of words $[w_0, w_1, w_2, ... , w_n]$ that form a sentence structure in the language. 
\end{definition}

An underlying assumption in research on text processing is that a text is more than the sum of its sentences. 
It is not sufficient to collect an arbitrary sequence of sentences in order to obtain a text. 
Sentences in a text are supposed to be related to one another to make a whole. 

\begin{definition}
A relationship function $R(s_i,s_j)$ indicates whether two given sentences in a text are related. 
The domain of this function is sentences in a text and its range is a binary value $
\lbrace 0,1\rbrace$. 
\end{definition} 

Having the above definition, relationships across all sentences in a text can be represented by a set $P$ containing relationships between any pairs of sentences in the text. 

\begin{definition}
Let $r_{ij}= R(s_i,s_j)$ indicates the relationship between a sentence pair $(s_i,s_j)$ in a text $T$, the set $P$ contains all $r_{ij}$ for any pair of sentences in $T$.
\end{definition}  

Although we define $P$ as a set but it can be partially structured, e.g.\ when sentence $s_i$ has to proceed sentence $s_j$ in a text. 


The distinction between texts with respect to their coherence is in the last resort a matter of degree \cite{haliday76}: Which one is more coherent? 
In order to rank texts with respect to their coherence, we need to capture patterns that frequently occur in well-written texts, and rarely in less coherent ones.  
Coherence patterns are template of relationships among sentences in texts, such that their frequencies assist to distinguish coherent texts from incoherent ones. 

\begin{definition}
A coherence pattern is a subset of relations $p \subseteq P$ occurring among sentences of a text.   
\end{definition}

We define a function to model how coherence patterns are extracted from a corpus of texts. 

\begin{definition}
Given a corpus $C$ which includes texts with different degrees of coherence, a pattern mining method $M$ extracts all subsets of relations that occur in texts in $C$. 
\end{definition} 

The output of the pattern mining process on a corpus of texts is a set of coherence patterns. 
The coherence of a text can be modeled by a vector of frequencies of these patterns in the text. 

\begin{definition}
Let $P=\lbrace p_0,p_1,p_2,...,p_m \rbrace$ be a set of extracted patterns from a corpus of texts $C$, the perceived coherence of each text $T$ is represented by a vector $\phi = <f_0, f_1, f_2,...,f_m>$, where $f_k$ is the frequency of pattern $p_k$ in text $T$. 
\end{definition}

Having vector representations of coherence lets us to employ machine learning models to rank texts. 
In Chapter \ref{ch:rel_work}, we describe different approaches to modeling relationships between a pair of sentences, i.e.\ $R(s_i,s_j)$, and how they represent the set of all relations in a text $P$.  
There, we also review how method $M$ is derived by different computational models. 
We employ a powerful representation method from the literature and develop our approach for extracting coherence patterns in Chapter \ref{ch:coh_patterns}. 
We further improve the predictive power of coherence patterns by a new approach to relationship representation among sentences $P$ in Chapter \ref{ch:lex_graph}. 

\section{The Linguistics of Coherence}

The aforementioned formal definition of the problem is sufficient for the research in this thesis to develop a representation of cohesive relations in a text, and extract coherence patterns. 
However, since coherence is a semantic property of a text, the definition needs to be related with linguistic properties of a text. 

Our aim is to develop an approach which provides a representation of the coherence of a text. 
Therefore we need to understand what aspects in sentences serve to relate sentences and how coherence patterns are modeled in linguistics of coherence. 
Therefore, we explain related linguistic properties that are required to complete our definition. 


\subsection{Text}

The first definition in our formal model of coherence is about text. 
In linguistic research the word ``text'' is used to refer to any passage, spoken or written, of whatever length that forms a unified whole \cite{halliday76}. 
In this thesis, we follow other coherence models \cite{barzilay08,guinaudeau13} and use the word ``text'' to denote a merely monologue written passage, which is more than one sentence.
One sentence texts, of course, do exist, like public notices, proverbs and advertising slogans and the like. 
An a sample text in Example \ref{ex:coh-one-sent-text}\footnote{Taken from \url{https://www.engvid.com/english-resource/50-common-proverbs-sayings/}, accessed 1 June 2018.}: 

\begin{examples}
    \label{ex:coh-one-sent-text}
    A journey of a thousand miles begins with a single step.
\end{examples}

However, due to texts consisting of one sentence only are fairly rare, we assume texts contains several sentences.  
We also assume that texts are written in a formal language, in contrast to informal like tweets\footnote{A post made on the social media application Twitter}. 

\subsection{Coherence}

Coherence is a crucial factor of a well-written text. 
It makes the text distinguishable from an unrelated sequence of sentences. 
We follow \newcite{stede12} and define coherence as a property of a text that is designed around a common topic. 
A coherent text discusses a sequence of topics in a structured way that a reader can recognize and relate to one another, and collectively render the text as a unified whole. 
Each topic is tending to occupy a (topic) segment of the text \cite{hearst97}.  
Coherence is defined based on the relations and structures of topic segments in a text. 
This structure is sometimes referred to as global, since it is coarse-grained and spans the entire text \cite{}.  
In general, however, defining the notion of topic, and recognizing topics and the boundaries between  text segments corresponding to them are not straight forward \cite{stede12} (see Chapter \ref{ch:rel-work} for computational related work). 

From the linguistic viewpoint, a coherent text employs linguistic devices, more-readily identifiable linguistic signals, to relate sentences of a text to each other. 
These devices signal readers to interpret each sentence considering its relations with other linked sentences \cite{vandijk77}. 
Therefore, understanding the text implies uncovering these relationships.  
The way that linguistic devices are used to relate sentences in a text is known as local coherence. 
In some literature in text linguistics \cite{halliday76}, this phenomena is referred to as cohesion.   
\newcite{stoddard91} (chapter 2) argues that these two are not distinguishable and can be used interchangeably. 
\newcite{stede12} stats that signals of local coherence serve as indicators of topic continuity, consequently the absent of a surface relation is a sign of topic shift. 
Since the research in this thesis is about modeling local coherence, henceforth in this thesis it is referred to as coherence modeling, unless this is not clear from the context. 

Prominent cohesive devices can be grouped in grammatical and lexical relations between elements of sentences \cite{halliday76}. 
Grammatical relations are reference, substitution, ellipsis and conjunctions. 
Lexical relations include any lexical-semantic relation, such as repetition, synonym, antonym, and the like, between words of sentences. 
Reference cohesive devices are also known as entity-based relations, with this intuition that related sentences in a text keep referring to the same named entity. 
The sample text in Example \ref{ex:coh-ref}\footnote{Taken from \url{https://web.stanford.edu/class/archive/cs/cs224n/cs224n.1162/handouts/cs224n-lecture11-coreference-6up.pdf}, on June 2 2018.} shows how entity-based relation link sentences. 

\begin{examples}
	\label{ex:coh-ref}
	\textbf{Mr.\ Obama} visited the city. \\
	\textbf{The president} talked about Milwaukee’s economy. \\
	\textbf{He} mentioned new jobs. \\
\end{examples} 


One of the popular entity-based frameworks for local coherence modeling is Centering Theory \cite{grosz95}. 
The main hypothesis of this theory is the center that at any given point in a text is most salient entity.  
The center captures the focus of attention in the reader mind \cite{grosz95}.
Centering Theory accounts of the process of flowing the center in a text. 
The original centering theory (for English texts) see the grammatical role as the most important linguistic sign of the text center. 
Specifically, the grammatical subject is taken the default position for the text center. 
Depending on the configuration of grammatical roles in adjacent sentence, \newcite{brennan87} define four different types of transitions for the text center. 
These transitions capture the smoothness of the center move from one sentence to another. 
When a latter sentence focuses on the topic, i.e.\ center, transition is Continue, the other types involve topic shift: Retain, Smooth Shift, and Rough Shift. 
Centering has been applied directly in models of coherence \cite{karamanis04}.  
However, since it requires human annotations for transitions between sentences, other models employ the basic Centering principles as soft constraints or features in a probabilistic framework such as the Entity Grid \newcite{barzilay05}, which we discuss in more detail in Chapter \ref{ch:rel-work}.

Further research shows that grammatical role information are far less predictive for tracking the text center in German, as a free word order language \cite{strube.acl96}. 
Instead the functional information structure \cite{danes74} are considered. 
This structure captures flows between given information vs. new information in a text. 
Some information in a sentence is known or old, because of it is explained by preceding sentences of the sentence, and some information are new.  
The functional information structure is explained by \newcite{halliday76} as theme-rheme relations in a text. 
Theme can be taken as given information and rheme is (almost) similar to new information. 
Thematic relations between sentences in texts reveal some patterns \cite{danes74}, which are called  coherence patterns.  
In Chapter \ref{ch:coh-patterns}, where we define our coherence patterns, we give more explanation about functional information structure \cite{danes74} in texts.  


The other perspective of local coherence is lexical cohesion, which cohesive effect based on lexical-semantic relations between words. 
An advantage of lexical cohesion, on contrary to entity-based models, is that it requires no annotation at all; they are accessible directly from the text. 
The insight of local coherence on the grounds of lexical relations is that content words in a text do not occur independently of one another, but bear semantic similarity because of common topic. 
A form of lexical cohesion is reiteration, which involves different type of lexical relations such as repeating a word, using a synonym of a word, employing a superordinate word, and using a general word. 
Example \ref{ex:coh-lex} is taken from \cite{halliday78} to illustrate these relations:

\begin{examples}
	\label{ex:coh-lex}
	(a) Repetition: There was a large \textbf{mushroom} growing near her, about the same height as herself; and when she had looked under it, it occurred to her that she might as well look and see what was on the top of it.\\
	She stretched herself up on tiptoe, and peeped over the edge of the \textbf{mushroom}, ... 

	(b) Synonymy: Accordingly ... I took leave, turned to the \textbf{ascent} of the peak. \\
	The \textbf{climb} is perfectly easy. 

	(c) Superordinate: Henry's bought himself a new \textbf{Jaguar}. \\
	He practically lives in the \textbf{car}. 

\end{examples} 

In (c) the word ``car'' is a superordinate, any word whose meaning includes that of the earlier one, of ``Jaguar'', as vehicle is a superordinate of the car, spoon of teaspoon, and the like. 
The boundary between the reiteration type of lexical cohesion and reference type in grammatical relations, i.e.\ entity-based relations, is by no means clearcut\footnote{However, properly speaking, reference is irrelevant to lexical cohesion. As we discuss later lexical cohesion relationship between two items does not need to imply a coreferent relationship.} \cite{halliday78}. 
This can slightly clarify the reason that the research in this thesis considers local coherence and cohesion almost synonym.

It is not necessary for two lexical occurrences to be coreferent, however, in order to for them to be cohesive. Consider Example \ref{ex:coh-nonref} from \newcite{halliday78}. 
In (a) word ``boy'' and  word ``boys'' are not coreferent, however, their semantic relationship serves to link the sentences. 
This relationship is not in any way dependent on the presence of other items; it is not the wriggling that provides the context, as (b) shows. 

\begin{examples}
	\label{ex:coh-nonref}
	Why does this little boy have to wriggle all the time? \\
	(a) Good boys don't wriggle. \\
	(b) Boys should be kept out of here. \\
\end{examples} 



Similarities between words are well-defined through a set of lexical-semantic relations, which show up in many implemented knowledge resources such as WordNet \cite{}. 


% % Lovejoy and Lance (1991), in their study of written discourse, show that cohesion can be achieved through the operation of theme-rheme. 
% This movement represents how information is managed.
% According to Lovejoy and Lance (1991), theme is ``the point of departure" for the representation of information" and rheme ``constitutes the information the writer wishes to impact about the theme".
% % These two elements are presented alternatively in a text to form a connected text. 
% while theme conveys information that is initially introduced in discourse, rheme presents specific 
% information regarding the theme. 


\section{Evaluation}

% As this movement continues, ideas in a text or discourse are expected to flow along smoothly and are easier for the reader to understand. 

% While old information (theme) is presented as background information in each statement, new information (rheme) is introduced to clarify information in the theme. 

% Morgen and Sellner (1980) emphasize that the cohesion and coherence are independent.  

% Carrel (1982) contends that cohesion does not bring about coherence, but believes that the cohesion is the effect of coherence not a cause of coherence. 
% From a textual perspective, Hoey (1991) examined how lexical cohesive elements would make a text organized. 
% He examined how lexical features and syntactic repetition would contribute to cohesion. 

% Within this general framework, cohesion is regarded as an element that accommodates coherence. 

% When a text is cohesive and coherence, it will enable the reader to process information more rapidly. 
% Hoey claims that ``cohesion is a property of the text and coherence is a facet of the reader's 
% evaluation of a text". 

% According to Hoey (1991) lexical connections as a major cohesive device constructs a matrix and creates a net of bonds in texts. 

% He proposes that lexical relations can show the relatedness of sentences within a text. 

% Johns (1986) divides coherence into two types: text-based and reader-based. 

% By her definition, text-based coherence refers to an inherent features of the text, which involves cohesion and unity. 
% This type of coherence involved how sentences are linked and how text is unified. 
% Reader-based coherence on the other hand requires successful interaction between the reader and the text. 

% In this type, coherence is based on the degree of compatibility between the reader's expectations and the intended meaning through the underlying structure of a text. 
% Connor and Johns (1990) describe coherent text ``as text in which the expectations of the reader are fulfilled".
% Readers use her knowledge of the world to interprets a text expecting that their knowledge will correspond to the organization and argument of a text. 
% The reader relies on this kind of knowledge to anticipate information that will be subsequently presented. 
% Interacting with readers, a coherent text accommodates the readers expectations of the order of ideas, contributing to the comprehension of the text. 
% By the same token, as logical ideas are presented through well-connected words and sentence the writer helps readers interpret and process information in a text more easily (Tannen, 1984).
% Lautamatti (1987) defines the term topic as what the sentence is about and the term comment as information about the topic. 
% All sentence topics are related in certain ways to the global discourse topic of the text. 
% The patterns of relations between discourse topics and subtopics are called topical development of discourse. 
% Grabe (1985) proposes the pragmatic function of coherence. 
% He identifies three features that are essential to coherence: a discourse theme, a set of relevant assertions relating logically among themselves. 
% The sentence topic in English is often correlates with grammatical subject and comment often correlates with the grammatical predicate, which bears the sentinel focus. 
% While patterns of theme and rheme connections can account for only some part of a text, diversity of patterns deal with an entire text. 
% The theory of cohesive ties introduced by Halliday and Hasan (1976) was modified into a theory of cohesive harmony (Hasan 1984, Haliday and Hasan, 1989). 
% Due to the limitations of the use of cohesive ties to analyze texts as coherent and well-written, Hasan (1984) formulated a new theory to account for the fact that cohesion contributes to coherence. 
% In her new approach, coherence is not determined by the type and quantity of cohesive ties that appear in a text, but it is mainly characterized by the degree and frequency of with which these ties interact with each other. 
% According to this theory, there are two cohesive ties which can interact with each others: those that form identity chain expressed with the use of co-referent entities, and those that form similarity strings expressed through substitution, ellipsis, repetition, synonymy, antonymy, hyponymy, and meronym. 
% Interaction does occur when one member of a string or chain is in the identical relationship to more than one member of another string or chain.

% An increasing number of researchers and practitioners in NLP face the prospect of havin to work with entire texts, rather than individual sentences. 
% While it is clear that text must have useful structure, its nature is less clear, making it more difficult to exploit in applications. 
% Discourse commonly comprises a sequences of sentence although it can be found even within a single sentence. 
% Within a discourse, the patterns formed by its sentences mean that the whole conveys more than the sum of its separate parts. 
% Another point  about discourse is that it exploits language  features, which allow speaker to specify that they are talking about something they have talked about before in the same discourse; indicating a realtion that holds between the states, events, beliefs, etc.  presented in the discourse; changing to a new topic or resuming one from earlier in the discourse.
% Discourse structures are the patterns that one sees in multi-sentence (multi-clausal) texts. 
% Recognizing these patterns in terms of the elements that compose them is essential to correctly deriving and interpreting information in the text. 
% The elements may be topic, each  about a set entities and what is being said about them.
% These type of relations are mainly called as coherence relations by Webber et al. (NLE 2012). 
% Discourse can be structured by its topics, each comprising a set of entities and a limited range of things being said about them. 
% Each topic may involve a set of entities, which may (but not have to) change from topic to topic. 
% This aspect of structure has been modeled as entity chains (Barzilay and Lapata, 2008).
% Patterns of entity chains can also be characteristic of particularly types of discourse, and therefore be of value in assessing the quality of automatically generated text. 
% Low-level evidence for the topic structure of discourse comes from the strong correlation between topic and lexical usage, which Halliday and Hasan (1976) call lexical cohesion. 
% Lexical chains can simply be a matter of the density of related terms within a segment, or of particular patterns of related terms, such as lexical chains (Barzilay and Elhadad 1997; Galley et al, 2003; Clarke and Lapata, 2010) defined as sequence of semantically related words. 


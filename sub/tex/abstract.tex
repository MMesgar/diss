\selectlanguage{english}
\addchap*{\abstractname}

Local coherence is a key property of well-written texts. It makes a multi-sentence text distinguishable from a sequence of randomly strung sentences. 
The task of local coherence modeling is about the way that sentences in a text link up one another.  
Solving this task is beneficial for assessing the quality of texts. 
Moreover, a coherence model can be integrated as a component in text generation systems such as text summarization. 

In this thesis, we present a graph-based approach to local coherence modeling that accounts for the connectivity structure among sentences in a text. 
Graphs give our model the capability to take into account relations between non-adjacent sentences as well as those between adjacent sentences. 
Moreover, the connectivity style among nodes in graphs reflects how sentences in a text are related to each other. 

We first employ entity graph approach, proposed by \newcite{guinaudeau13}, to represent a text via a graph. 
In this graph representation of a text, nodes encode sentences and edges depict existence of a pair of coreferent mentions in sentences.    
Considering such graph representations of texts, we introduce graph-based features to encode the structure of edges in a graph, and therefore the coherence property of the corresponding text to the graph.   
We extract all subgraphs in graphs and refer to them as coherence patterns. 
The diversity between the frequencies of coherence patterns in graph representations of texts captures the differences in the structure of graphs.  
Consequently, the frequencies of coherence patterns correlate with the perceived coherence of texts.  

In order to complete our approach to coherence modeling, we propose a new graph representation of texts, instead of the entity graph.  
Our approach employs lexical relations among words in sentences, instead of only coreference relations, to model semantic relations between sentences as a graph. 
The combination of our method for coherence pattern mining and our lexical approach to text representation yields a complete coherence model. 

We mainly evaluate our approach on the readability assessment task because a key factor for this task is coherence. 
In a coherent text, sentences are attached together to unify the text as a whole. 
Such texts are easy to read and consequently need less effort from their readers. 
By extensive experiments on two separate readability assessment datasets, 
we investigate the correlation between frequencies of coherence patterns in texts and readability ratings assigned by human judges.    
We further use a machine learning method to learn how coherence patterns can rank texts with respect to readability. 
Since the ultimate goal of coherence models is to be used in text generation systems, we investigate how our coherence patterns can be integrated into a graph-based text summarizer to produce informative and coherent summaries. 
An implementation of the frameworks discussed in this thesis is publicly available.

\selectlanguage{ngerman}
\addchap*{\abstractname}

Lokale Kohärenz ist eine Schlüsseleigenschaft von gut geschriebenen Texten. 
Sie unterscheidet einen Text mit mehreren Sätzen von einer Folge zufällig aufgereihter Sätze. 
Bei der lokalen Kohärenzmodellierung geht es darum, wie Sätze in einem Text miteinander verbunden sind.
Sie ist nützlich für die Bewertung von Textqualität, aber auch als Bestandteil von Texterzeugungssystemen wie z.B. für die Textzusammenfassung.

In dieser Doktorarbeit präsentieren wir einen graphbasierten Ansatz zur lokalen Kohärenzmodellierung, welcher die Verbindungsstruktur unter den Sätzen in einem Text darstellt. 
Die Graphen geben unserem Modell die Fähigkeit, sowohl die Verhältnisse zwischen benachbarten als auch zwischen nicht benachbarten Sätzen zu berücksichtigen.
Darüber hinaus spiegelt der Verbindungsstil unter Knoten in Graphen, wie sich die Sätze in einem Text aufeinander beziehen. 
Zuerst beschäftigen wir uns mit dem von \newcite{guinaudeau13} entwickelten \mbox{Entity-Graph-Ansatz}.
In diesem Ansatz werden Sätze durch Knoten repräsentiert, und Kanten zwischen Knoten repräsentieren koreferente Ausdrücke in zwei Sätzen.
Unter Berücksichtigung solcher grafischen Darstellungen von Texten führen wir graphbasierte Funktionen ein, um die Strukturen der Kanten in einem Graph und folglich die Kohärenzeigenschaft des entsprechenden Textes zu kodieren. 

Wir extrahieren alle Untergraphen in Graphen und bezeichnen sie als Kohärenzmuster. 
Die Variabilität der Häufigkeiten der Kohärenzmuster in den Graphdarstellungen erfasst die Unterschiede zwischen den Strukturen der Graphen. 
Folglich korrelieren die Häufigkeiten der Kohärenzmuster mit der wahrgenommenen Kohärenz von Texten.
Um unseren Ansatz zur Kohärenzmodellierung zu vervollständigen, schlagen wir, als Alternative zum Entity-Graph-Ansatz, eine neue Graphrepräsentation von Texten vor. 
Unser Ansatz nutzt lexikalische Beziehungen zwischen Wörtern, und nicht nur Koreferenzbeziehungen, um semantische Beziehungen zwischen Sätzen als Graph zu modellieren. 
Die Kombination unserer Methode für die Kohärenzmusterextraktion mit unserem lexikalischen Ansatz zur Textrepräsentation ergibt ein vollständiges Kohärenzmodell.
Wir evaluieren unseren Ansatz überwiegend im Hinblick auf die Lesbarkeitsbewertung von Texten, weil Textkohärenz ein Schlüsselfaktor für diese Aufgabe ist. 
In einem kohärenten Text werden Sätze zu einem Ganzen zusammengefügt. 
Derartige Texte sind einfach zu lesen und zu verstehen und erfordern folglich weniger Aufwand von ihren Lesern.
Durch umfangreiche Versuche auf zwei verschiedenen Datensätzen zur Lesbarkeitsbewertung untersuchen wir die Korrelation zwischen den Häufigkeiten der Kohärenzmuster in
Texten und von menschlichen Subjekten vorgenommenen Lesbarkeitsbewertungen.

Wir verwenden zusätzlich ein maschinelles Lernverfahren, um zu lernen, wie Kohärenzmuster Texte in Bezug auf ihre Lesbarkeit ordnen können. 
Da das eigentliche Ziel der Kohärenzmodellierung der Einsatz in Texterzeugungssystemen ist, untersuchen wir, wie unsere Kohärenzmuster
in ein graphbasiertes Textzusammenfassungssystem zum Erzeugen von informativen und kohärenten Zusammenfassungen integriert werden können.
Eine Implementierung der in dieser Doktorarbeit diskutierten Ansätze ist öffentlich verfügbar.





\selectlanguage{english}
\addchap*{\abstractname}

Coherence is an essential property of well-written texts. 
It distinguishes a multi-sentence text from a sequence of randomly strung sentences. 
The task of local coherence modeling is about the way that sentences in a text link up one another. 
Solving this task is beneficial for assessing the quality of texts. 
Moreover, a coherence model can be integrated into text generation systems such as text summarizers to produce coherent texts. 

In this dissertation, we present a graph-based approach to local coherence modeling that accounts for the connectivity structure among sentences in a text. 
Graphs give our model the capability to take into account relations between non-adjacent sentences as well as those between adjacent sentences. 
Besides, the connectivity style among nodes in graphs reflects the relationships among sentences in a text. 

We first employ the entity graph approach, proposed by \newcite{guinaudeau13}, to represent a text via a graph. 
In the entity graph representation of a text, nodes encode sentences and edges depict the existence of a pair of coreferent mentions in sentences. 
We then devise graph-based features to capture the connectivity structure of nodes in a graph, and accordingly the connectivity structure of sentences in the corresponding text. 
We extract all subgraphs of entity graphs as features which encode the connectivity structure of graphs.    
Frequencies of subgraphs correlate with the perceived coherence of their corresponding texts. 
Therefore, we refer to these subgraphs as coherence patterns. 

In order to complete our approach to coherence modeling, we propose a new graph representation of texts, rather than the entity graph. 
Our approach employs lexico-semantic relations among words in sentences, instead of only entity coreference relations, to model relationships between sentences via a graph. 
This new lexical graph representation of texts plus our method for mining coherence patterns make our coherence model. 

We evaluate our approach on the readability assessment task because a primary factor of readability is coherence. 
Coherent texts are easy to read and consequently demand less effort from their readers. 
Our extensive experiments on two separate readability assessment datasets show that frequencies of coherence patterns in texts correlate with the readability ratings assigned by human judges. 
By training a machine learning method on our coherence patterns, our model outperforms its counterparts on ranking texts with respect to their readability. 
As one of the ultimate goals of coherence models is to use them in text generation systems, we show how our coherence patterns can be integrated into a graph-based text summarizer to produce informative and coherent summaries. 
Our coherence patterns improve the performance of the summarization system based on both standard summarization metrics and human evaluations. 
An implementation of the approaches discussed in this dissertation is publicly available\footnote{\url{https://github.com/MMesgar/}}. 

\selectlanguage{ngerman}
\addchap*{\abstractname}

Kohärenz ist eine wesentliche Eigenschaft von gut geschriebenen Texten. 
Sie unterscheidet einen Text mit mehreren Sätzen von einer Folge von zufällig aufgereihten Sätzen. 
Die Aufgabe der lokalen Kohärenzmodellierung geht es darum, wie Sätze in einem Text miteinander verbunden sind. 
Die Lösung dieser Aufgabe ist  zur nützlich für die Bewertung von Textqualität. 
Außerdem kann ein Kohärenzmodell in Textgenerierungssystemen wie z.B.\ Textzusammenfassungssysteme integriert werden, um zusammenhängende Texte zu erzeugen. 

In dieser Doktorarbeit präsentieren wir einen graphbasierten Ansatz zur lokalen Kohärenzmodellierung, welcher die Verbindungsstruktur unter den Sätzen in einem Text darstellt. 
Die Graphen geben unserem Modell die Fähigkeit, sowohl die Verhältnisse zwischen benachbarten als auch zwischen nicht benachbarten Sätzen zu berücksichtigen. 
Darüber hinaus spiegelt der Verbindungsstil unter Knoten in Graphen die Beziehungen zwischen den Sätzen in einem Text wider. 

Zuerst verwenden wir den von Guinaudeau und Strube (2013) entwickelten Entity-Graph-Ansatz, um einen Text über einem Graph darzustellen. 
In diesem Ansatz werden Sätze durch Knoten repräsentiert, und Kanten zwischen Knoten repräsentieren koreferente Ausdrücke in zwei Sätzen. 
Danach entwickeln wir Graph-basierte Eigenschaften zum Erfassen der Verbindungsstruktur von Knoten in einem Graph, und von den Sätzen im dazugehörigen Text. 
Wir extrahieren alle Untergraphen der Entity-Graphen als Merkmale, die die Verbindungsstruktur der Graphen repräsentieren.  
Die Häufigkeit der Untergraphen korrelieren mit der wahrgenommenen Kohärenz ihrer entsprechenden Texte. Deshalb beziehen wir uns auf diese Untergraphen als Kohärenzmuster. 

Um unseren Ansatz zur Kohärenzmodellierung zu vervollständigen, schlagen wir, als Alternative zum Entity-Graph eine neue Graphrepräsentation von Texten vor. 
Unser Ansatz nutzt lexico-semantische Beziehungen zwischen Wörtern, und nicht nur Koreferenzbeziehungen, um semantische Beziehungen zwischen Sätzen als Graf zu Modellieren.  
Diese neue lexikalische Graphrepräsentation von Texten plus unsere Methode für die Kohärenzmusterextraktion bildet unser Kohärenzmodell. 

Wir evaluieren unseren Ansatz überwiegend im Hinblick auf die Lesbarkeitsbewertung von Texten, weil Textkohärenz ein Schlüsselfaktor für diese Aufgabe ist. 
Kohärente Texte sind einfach zu lesen und zu verstehen und erfordern folglich weniger Aufwand von ihren Lesern. 
Durch umfangreiche Versuche auf zwei verschiedenen Datensätzen zur Lesbarkeitsbewertung untersuchen wir die Korrelation  zwischen den Häufigkeiten der Kohärenzmuster in Texten und von menschlichen Subjekten vorgenommenen Lesbarkeitsbewertungen.
 
Durch die Schulung eines maschinellen Lernverfahrens auf unsere Kohärenzmuster übertreffen unser Modell seine Gegenstücke im Rangierung der texten hinsichtlich ihrer Lesbarkeit. 
Da eines der eigentlichen Ziele der Kohärenzmodellierung der Einsatz in Texterzeugungssystemen ist, zeigen wir, wie unsere Kohärenzmuster in ein graphbasiertes Textzusammenfassungssystem zum Erzeugen von informativen und kohärenten Zusammenfassungen integriert werden können. 
Unsere Kohärenzmuster verbessern die Leistung des Zusammenfassungssystems basierend auf beide Standardzusammenfassungsmetriken und menschliche Bewertungen.
Eine Implementierung der in dieser Doktorarbeit diskutierten Ansätze ist öffentlich verfügbar\footnote{\url{https://github.com/MMesgar/}}. 

\selectlanguage{english}
\addchap*{\abstractname}

Local coherence is a key property of well-written texts. It makes a multi-sentence text distinguishable from a sequence of randomly strung sentences. 
The task of local coherence modeling is about the way that sentences in a text link up one another.  
Solving this task is beneficial for assessing the quality of texts. 
Moreover, a coherence model can be integrated as a component in text generation systems such as text summarization. 

In this thesis, we present a graph-based approach to local coherence modeling that accounts for the connectivity structure among sentences in a text. 
Graphs give our model the capability to take into account relations between non-adjacent sentences as well as those between adjacent sentences. 
Moreover, the connectivity style among nodes in graphs reflects how sentences in a text are related to each other. 

We first employ entity graph approach, proposed by \newcite{guinaudeau13}, to represent a text via a graph. 
In this graph representation of a text, nodes encode sentences and edges depict existence of a pair of coreferent mentions in sentences.    
Considering such graph representations of texts, we introduce graph-based features to encode the structure of edges in a graph, and therefore the coherence property of the corresponding text to the graph.   
We extract all subgraphs in graphs and refer to them as coherence patterns. 
The diversity between the frequencies of coherence patterns in graph representations of texts captures the differences in the structure of graphs.  
Consequently, the frequencies of coherence patterns correlate with the perceived coherence of texts.  

In order to complete our approach to coherence modeling, we propose a new graph representation of texts, instead of the entity graph.  
Our approach employs lexical relations among words in sentences, instead of only coreferent relations, to model semantic relations between sentences as a graph. 
The combination of our method for coherence pattern mining and our lexical approach to text representation yields a complete coherence model. 

We mainly evaluate our approach on the readability assessment task because a key factor for this task is coherence. 
In a coherent text, sentences are hanged together to unify the text as a whole. 
Such texts are easy to read and understand, and consequently need less effort from their readers. 
By extensive experiments on two separate readability assessment datasets, 
we investigate the correlation between frequencies of coherence patterns in texts and readability ratings assigned by human judges.    
We further use a machine learning method to learn how coherence patterns can rank texts with respect to readability. 
Since the ultimate goal of coherence models is to be used for text generation systems, we investigate how our coherence patterns can be integrated into a graph-based text summarizer to produce informative and coherent summaries. 
An implementation of the frameworks discussed in this thesis is publicly available.

\selectlanguage{ngerman}
\addchap*{\abstractname}

Die lokale Kohärenz ist eine Schlüsseleigenschaft von den gut geschriebenen Texten. Es macht einen Mehrsatztext unterschiedlich von einer Folge der zufällig aufgereihten Sätzen.
Die Aufgabe der lokalen Kohärenzmodellierung ist nämlich ein Weg, dadurch die Sätze in einem Text miteinander verbinden.
Die Lösung dieser Aufgabe ist nützlich für die Bewertung der Textqualität.
Außerdem kann eine Kohärenzmodellierung als der Bestandteil der Texterzeugungssysteme Beispiel die Textzusammenfassung integriert werden.

In dieser Doktorarbeit präsentieren wir einen graphbasierten Ansatz der lokalen Kohärenzmodellierung, welcher die Verbindungsstruktur unter den Sätzen in einem Text darstellt.
Die Graphen geben unserem Modell die Fähigkeit, sowohl die Verhältnisse zwischen den nicht angrenzenden Sätzen als auch die Verhältnisse zwischen diesen und benachbarten Sätzen zu berücksichtigen. 
Darüber hinaus spiegelt der Verbindungsstil unter Knoten in Graphen, wie sich die Sätze in einem Text aufeinander beziehen.
Zuerst widmen wir uns mit dem Wesen Graph-Ansatz, vorgeschlagen von \newcite{guinaudeau13}, um einen Text über den Graph darzustellen.
In diesem Graph erwähnt in den Sätzen die Darstellung eines Textes, die Knoten, die die Sätze  Kodieren und die Kanten, die die Existenz eines Paares von Korefrenten abbilden.
Unter Berücksichtigung solcher grafischen Darstellungen von Texten führen wir graphbasierte Funktionen ein, um die Strukturen der Kanten in einem Graph und folglich die Kohärenzeigenschaft des entsprechenden Textes zu der Graphen zu kodieren.
Wir extrahieren alle Untergraphen in Graphen und verweisen auf sie als Kohärenzmuster.
Die Vielfalt zwischen den Frequenzen der Kohärenzmuster in den Graphdarstellungen von Texten erfasst die Unterschiede zwischen Strukturen von Graphen.
In der Folge korrelieren die Frequenzen der Kohärenzmuster mit der wahrgenommenen Kohärenz von Texten.

Um unseren Ansatz an die Kohärenzmodellierung zu vervollständigen, schlagen wir ein neues Graphrepresentation der Texte vor, anstatt des Entitätgraphes.
Unser Ansatz setzt lexikalische Beziehungen unter Wörtern in Sätzen statt nur Koreferentbeziehungen ein, um semantische Beziehungen zwischen Sätzen sowie ein Graph zu modellieren.
Die Kombination unserer Methode für Kohärenzmusterverminung und unserem lexikalischen Ansatz
zur Textrepräsentation ergibt sich ein vollständiges Kohärenzmodell.

Wir evaluieren unseren Ansatz überwiegend auf die Lesbarkeitsbewertung der Aufgabe, weil ein Schlüsselfaktor für diese Aufgabe Kohärenz ist.
In einem zusammenhängenden Text werden Sätze zur Vereinigung des Textes als Ganzes zusammengehängt.
Solche Texte sind einfach zu lesen und zu verstehen und folglich benötigen sie weniger Bemühung von der Seite ihren Lesern.
Durch weitgehende Versuche auf zwei getrennte Lesbarkeitsbewertung der Datensätze untersuchen wir die Korrelation zwischen Frequenzen der Kohärenzmuster in Texten und Lesbarkeitsbewertungen, die von menschlichen Richtern zugeteilt sind.
Wir benutzen zusätzlich ein maschinelles Lernverfahren zu lernen, wie Kohärenzmuster die Texte in Bezug auf Lesbarkeit rangieren können.
Seit dem höchsten ziel der Kohärenzmodelle ist für die Texterzeugungssysteme zu verwenden, untersuchen wir, wie unsere Kohärenzmuster in einem graphbasierten Textzusammenfasser zum Erzeugen der informativen und kohärenten Zusammenfassungen integriert werden können. 
Eine Implementierung der in dieser Doktorarbeit diskutierten Frameworke ist öffentlich verfügbar.




